%
% Phantom space for abbreviations
%
\usepackage{xspace}
%
% To insert doi identifiers
%
\usepackage{doi}
\renewcommand{\doitext}{DOI }
%
% Improve citations from biblio
%
\usepackage{cite}
%
% This is to create hyperlinks for index, URLs and citations
% (now we can use the command \url{...} to create URL with hyperlink)
% 
%\usepackage{color}
%\usepackage[a4paper,colorlinks=true,urlcolor=blue,citecolor=blue,linkcolor=blue,bookmarks=false]{hyperref}
%
% For pasting text files
%
\usepackage{fancyvrb}
%
% Used to express formulas like n^th
%
%\usepackage{mathtools}
%
% For table formatting
%
%\usepackage{longtable}
%\usepackage{makecell}
%\usepackage{multirow}
%
% For plotting results
%
\usepackage{pgfplots}
\pgfplotsset{compat=newest}
%
% For placing floats
%
%\usepackage{placeins}
%
% For subfigure environment
%
\usepackage{subcaption}
\usepackage{cleveref}
\captionsetup[subfigure]{subrefformat=simple,labelformat=simple}
\renewcommand\thesubfigure{(\alph{subfigure})}
%
% For Appendix section
%
%\usepackage[titletoc,toc,title]{appendix}
%
% Definition of margins
%
%\usepackage[top=2cm,bottom=2cm,left=2cm,right=2cm]{geometry}
%
% Paragraph skip and indent
%
%\setlength\parskip{\medskipamount}
%\setlength\parindent{0pt}
%
% For itemize and enumerate spacing
%
\usepackage{enumitem}
%
% For International System of Units (SI)
%
\usepackage[binary-units]{siunitx}
%\sisetup{per-mode=symbol} % 1/s instead of s^-1
%
% For pseudo code
%
\usepackage{algorithm}
\usepackage{algpseudocode}
\usepackage{caption}
%
% For programming code
%
\usepackage{xcolor}
\usepackage{listings}
\lstdefinestyle{c}{
  belowcaptionskip=1\baselineskip,
  breaklines=true,
  xleftmargin=\parindent,
  language=C,
  showstringspaces=false,
  basicstyle=\footnotesize\ttfamily,
  keywordstyle=\bfseries\color{green!40!black},
  commentstyle=\itshape\color{black!60!},
  identifierstyle=\color{blue},
  stringstyle=\color{orange},
}
\lstset{style=c} % Default style


% Some page parameters
\setlength{\parskip}{\medskipamount}
% Horizontal rule
\newcommand{\HRule}{\rule{\linewidth}{0.2mm}}


%
% Frequently used abbreviations.
% - example: \ie this is an example
%
\def\eg{e.g.\xspace}
\def\ie{i.e.\xspace}
\def\mychap#1{Chapter~\ref{#1}\xspace}
\def\mysec#1{Section~\ref{#1}\xspace}
\def\myfig#1{Fig.~\ref{#1}\xspace}
\def\mytab#1{Tab.~\ref{#1}\xspace}
\def\myalg#1{Algorithm~\ref{#1}\xspace}
\newcommand{\ltx}{\LaTeX\xspace}
\newcommand{\txw}{TeXworks\xspace}
\newcommand{\mik}{MikTex\xspace}
\newcommand{\html}{HTML\xspace}
\newcommand{\xhtml}{XHTML\xspace}
\newcommand{\cmd}[1]{\texttt{#1}\xspace}

% Styles
\newcommand{\itemname}{\textbf}
\newcommand{\thead}{\textbf}

% To cite RFC, es. \rfc{822}
\newcommand{\rfc}[1]{RFC-#1\xspace}
% To cite file (es. \file{autoexec.bat}) or fake URI (i.e. \file{http://www.lioy.it/})
% for real URIs use \url o \href
\newcommand{\file}[1]{\texttt{#1}\xspace}
% For inline code
\newcommand{\code}[1]{\lstinline|#1|}
% Backslash
\newcommand{\bs}{\textbackslash}
% Term definition with insertion into the index
\newcommand{\tdef}[1]{\textit{#1}\index{#1}}
% Meta-term
\newcommand{\meta}[1]{\textit{#1}}
