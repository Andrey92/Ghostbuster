\chapter{Pin Control Protection Design}
\label{chap:design}

{\color{red}NB: Just draft chapter.}

In this chapter we first analyse Pin Control Attack in more detail, and then show the design and the implementation of the proposed defense.

\section{Pin Control Attack Analysis}

First, we try to better classify it in order to better understand which type of attack we are going to address.

We assert that our attack cannot be completely categorised using the taxonomy proposed in \cite{taxonomy}, resulting in at least two dimensions with \emph{Unknown} value.
\begin{enumerate}
	\item \itemname{Precondition}: we have at least two different variants of Pin Control Attack, which will be discussed in more detail in \chap \ref{chap:design},
		and they need different precondition. Both need at least \emph{local or remote access} to the system, and an a priori knowledge of the physical process.
	\item \itemname{Vulnerability}: in our case the exploited vulnerability is basically the lack of hardware interrupts as reaction to hardware configuration changes.
		This is about the communication between hardware and operating system and, to the best of our knowledge, it has never been addressed before.
	\item \itemname{Target}: the target layer resides between hardware and firmware/OS. They suggest the generic \emph{device} target for this case.
	\item \itemname{Attack method}: our attack installs a \emph{malware} inside the PLC. More specifically it is a rootkit that could not be detected
		with the current available protections, if any.
	\item \itemname{Effect}: our most dangerous effects are not about code execution, integrity violation or denial of service as for the most of the existing attacks.
		Our main effects are directly propagated into the physical world surrounding the violated system, as you may notice from example in \chap \ref{chap:intro},
		and they strongly depends on which type of process the PLC is actually controlling.
\end{enumerate}

The classification of Pin Control Attack based on the taxonomy proposed by Papp et al. \cite{taxonomy} is summarised in \mytab{\ref{tab:attack_class}}.

\begin{table}[h]
\begin{center}
{\renewcommand{\arraystretch}{1.5}
\begin{tabular}{|c|c|c|c|c|}
\hline
\multicolumn{5}{|c|}{\thead{Pin Control Attack}} \\
\hline
\textit{Precondition} & \textit{Vulnerability} & \textit{Target} & \textit{Attack method} & \textit{Effect}\\
\hline
Local or Remote Access, Miscellaneous & Unknown & Device & Malware & Unknown \\
\hline
\end{tabular}
\caption{Pin Control Attack classification based on Papp et al. \cite{taxonomy} taxonomy.\label{tab:attack_class}}
}
\end{center}
\end{table}

TODO Continue analysis.


\section{Pin Control Protection}

TODO Pin Control Protection Design.
