\chapter{Introduction}
\label{chap:intro}

Embedded systems are widely used today in various applications, from consumer such as cars, cell phones, home automation, to critical infrastructures
like power plants and power grids, water, gas or electricity distribution systems and production systems for food and other products.
Within the context of an Industrial Control System (ICS), these systems are known as PLCs (Programmable Logic Controllers).

But security for these systems is an open question and could be a more difficult long-term problem than security for desktop and enterprise computing,
both for their limited capabilities and for the physical side effects a security breach could lead to, including property damage, personal injury, death and
even environmental or nuclear disaster.

The PLCs control the outside world via their I/O interfaces: therefore, they must be both reliable and secure.
Digging into their architecture, we know that the I/O interfaces of PLCs (e.g., GPIO, SCI, JTAG, etc.),
are usually controlled by a so called System on a Chip (SoC), an integrated circuit that combines multiple I/O interfaces.
In turn, the pins in a SoC are managed by a pin controller, a subsystem of SoC, through which one can configure the operating mode of the pins, such as the input or output mode.
One of the most peculiar aspects of a pin controller is that its behavior is determined by a set of registers: by altering these registers one can change the behavior
of the chip in a dramatic way. In \cite{ghostplc}, Abbasi et al. showed how this feature is exploitable by attackers, who can tamper with
the integrity or the availability of legitimate I/O operations, factually changing how a PLC interacts with the outside world.

Based on these observations, they introduced a novel attack technique against PLCs, called Pin Control Attack.
The salient features of this new class of attacks are:
\begin{enumerate}
	\item It is intrinsically stealth. The alteration of the pin configuration does not generate any interrupt, preventing the Operating System (OS) to react to it.
	\item It is entirely different in execution from traditional techniques such as manipulation of kernel structures or system call hooking, which are typically
		monitored by anti-rootkit protection systems.
	\item It is viable. It is possible to build concrete attack using it.
\end{enumerate}

To overcome this sophisticated attack, we propose a defensive mechanism to extend the existing Linux Kernel Pin Control Subsystem to be able to detect our novel attack.
It is a challenging task for two reasons:
\begin{enumerate}
	\item The Pin Control lacks hardware interrupts, so is not possible to directly react to any configuration change. More complex detection mechanisms are needed
		in order to achieve the highest possible detection rate.
	\item The resources available within an embedded system like a PLC are extremely limited. Therefore, our solution must be extremely agile and light
		since the smallest delay in the PLC I/O operation can have unintended consequences for the controlled process.
\end{enumerate}

In \chap \ref{chap:related} we show how Pin Control Attack is different from the majority of attacks in the literature, and discuss the protection mechanisms currently available
for embedded systems, showing that no one of these is actually capable of detecting the attack.
In \chap \ref{chap:design} we present the architecture of our proposed Pin Control Protection, then we describe the implemented modules and provide an user manual.
The \chap \ref{chap:results} provides the results obtained during the experiments, showing the detection rate and the performance overhead on a PLC environment.
Finally, in \chap \ref{chap:conclusions} we analyse the shortcomings of our defense and the possible future works and improvements.
